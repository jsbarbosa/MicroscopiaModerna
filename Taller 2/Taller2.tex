%\documentclass[11pt]{article}

\documentclass[addpoints,10pt]{exam}

\usepackage[margin=2.5cm]{geometry}
\usepackage{graphicx}
\usepackage{listings}
\usepackage{xpatch}
\usepackage{color}
\usepackage{amsmath}

\makeatletter
\xpretocmd{\item@points@pageinfo}{\normalfont}{}{}
\xapptocmd{\item@points@pageinfo}{\bfseries}{}{}
\makeatother

\begin{document}
	\begin{center}
		\LARGE\scshape{Taller 2}
		
		\vspace{1cm}
		\large\scshape{Juan Barbosa - 201325901}
	\end{center}

	\begin{questions}
		{
			\question
			A white floodlight beam crosses a large volume containing a tenuous molecular gas mixture of mostly oxygen and nitrogen. Compare the relative amount of scattering occurring for the yellow (580 nm) component with that of the violet (400 nm) component.
		}
		
		De Rayleigh se sabe que el scattering es proporcional a $1/\lambda^4$:
		\begin{equation}
			S \propto \dfrac{1}{\lambda^4}
		\end{equation}
		
		Entonces el cociente entre el scattering del violeta $(5v)$ y el amarillo $(y)$:
		\begin{equation}
			\dfrac{S_v}{S_y} = \dfrac{1/\lambda_v^4}{1/\lambda_y^4} = \left(\dfrac{\lambda_y}{\lambda_v}\right)^4
		\end{equation}
		\begin{equation}
			S_v = S_y\left(\dfrac{\lambda_y}{\lambda_v}\right)^4 \approx 4.42S_y
		\end{equation}
		
		{
			\question
			A very narrow laserbeam is incident at an angle of 58" on a horizontal mirror. The reflected beam strikes a wall at a spot 5.0 m away from the point of incidence where the beam hit the mirror. How far	horizontally is the wall from that point of incidence?
		}
		
		El \'angulo incidente debe ser igual al reflejado. Lo cual implica que existe un tri\'angulo rect\'angulo con hipotenusa 5.0 m.
		
		\begin{equation}
		\sin(58^\circ) = \dfrac{d}{5.0} \qquad \longrightarrow d = 5\sin(58^\circ)
		\end{equation}
		
	\end{questions}
	
	
\end{document}
